\documentclass[12pt]{article}
\usepackage[english]{babel}
\usepackage{natbib}
\usepackage{url}
\usepackage[utf8x]{inputenc}
\usepackage{amsmath}
\usepackage{graphicx}
\usepackage{subfig}
\graphicspath{{images/}}
\usepackage{parskip}
\usepackage{fancyhdr}
\usepackage{vmargin}
\usepackage{algorithm}
\usepackage{amssymb}
\usepackage{csvsimple}
\usepackage{hyperref} 
\usepackage[noend]{algpseudocode}
\newcommand{\var}{\texttt}
\usepackage[toc,page]{appendix}

\setmarginsrb{3 cm}{2.5 cm}{3 cm}{2.5 cm}{1 cm}{1.5 cm}{1 cm}{1.5 cm}

\title{Heuristics \& Approximations Algorithms}								% Title
\author{Alexander Falk}								% Author
\date{13 April 2020}											% Date

\makeatletter
\let\thetitle\@title
\let\theauthor\@author
\let\thedate\@date
\makeatother

\pagestyle{fancy}
\fancyhf{}
\rhead{\theauthor}
\lhead{\thetitle}
\cfoot{\thepage}

\makeatletter
\let\OldStatex\Statex
\renewcommand{\Statex}[1][3]{%
  \setlength\@tempdima{\algorithmicindent}%
  \OldStatex\hskip\dimexpr#1\@tempdima\relax}
\makeatother

\begin{document}

%%%%%%%%%%%%%%%%%%%%%%%%%%%%%%%%%%%%%%%%%%%%%%%%%%%%%%%%%%%%%%%%%%%%%%%%%%%%%%%%%%%%%%%%%

\begin{titlepage}
	\centering
    \vspace*{0.5 cm}
    \includegraphics[scale = 0.5]{SDU_logo.png}\\[1.0 cm]	% University Logo
    \textsc{\LARGE University of Southern Denmark}\\[2.0 cm]	% University Name
	\textsc{\Large DM852}\\[0.5 cm]				% Course Code
	\textsc{\Large \thedate}\\[0.5 cm]				% Date
	\rule{\linewidth}{0.2 mm} \\[0.4 cm]
	{ \huge \bfseries \thetitle}\\
	\rule{\linewidth}{0.2 mm} \\[1.5 cm]
	
	\begin{minipage}{0.4\textwidth}
		\begin{flushleft} \large
			\emph{Submitted To:}\\
			Marco Chiarandini\\
            Lene Monrad Favrholdt \\
			IMADA \\
			Institute of Mathematics \& Computer Science\\
			\end{flushleft}
			\end{minipage}~
			\begin{minipage}{0.4\textwidth}
            
			\begin{flushright} \large
			\emph{Submitted By:} \\
			Alexander Lerche Falk\\
            Spring - Master of Computer Science\\
		\end{flushright}
        
	\end{minipage}\\[2 cm]
	
	
    
    
    
    
	
\end{titlepage}

%%%%%%%%%%%%%%%%%%%%%%%%%%%%%%%%%%%%%%%%%%%%%%%%%%%%%%%%%%%%%%%%%%%%%%%%%%%%%%%%%%%%%%%%%

\tableofcontents
\pagebreak

%%%%%%%%%%%%%%%%%%%%%%%%%%%%%%%%%%%%%%%%%%%%%%%%%%%%%%%%%%%%%%%%%%%%%%%%%%%%%%%%%%%%%%%%%

\section{Introduction}

The goal of this project is to show the knowledge acquired by the student in regards to Heuristic- and Local Search Algorithms. The algorithms are developed to suit the Capacitated Vehicle Routing Problem (CVRP). CVRP is a combinatorial optimization problemm, where a set of n customers has to be visited by m vehicles. Each customer can only be visited one vehicle. The sum of demands by the customers assigned to a vehicle must not overseed the capacity of the vehicle. All vehicles start and end their route at the depot (the starting point). The objective is to minimise the travelling cost and try to use as few vehicles as possible.  
Heuristics and Local Search algorithms are applied to CVRP, where the heuristics are laying the foundation of possible solutions of routes, and the Local Search algorithms try to improve the solutions.  
Three heuristic algorithms have been developed:  
\begin{itemize}
	\item Nearest Neightbour - Pick nearest customer from the depot or current customer without breaking the capacity
	\item Furthest Neightbour (with nearest clustering) - Pick customer furthest away from the depot, find the closest customers until the capacity is reached, and generate a route by using Nearest Neightbour
	\item K-Nearest Neightbour - Pick k nearest customers and randomly pick one of the k options. Proceed until no more customers without breaking the capacity
\end{itemize}  
Two local search algorithms have been developed:  

\begin{itemize}
	\item Two Opt - Pick a route generated by some heuristic. Pick two edges, swap the edges, and compare the distance from before and after the swap. If better, make the swap. 
	\item Three Opt - Pick a route generated by some heuristic. Pick three edges, swap the edges, and compare the distance from before and after the swap. If better, make the swap. 
\end{itemize}

At the end of the report, comparison of the heuristics and results are presented.
All the code can be found at \href{https://github.com/AlexanderFalk/2020_Project01_CS_HA}{GitHub}
\newpage

\section{Nearest Neightbour} 

The Nearest Neightbour algorithm is a simple, yet powerful, approach to tackle CVRP. The algorithm goes starts at the depot and picks the nearest customer by iterating through all of the customers. It marks the customer as visited, adds it to a vehicle, and continues until the capacity of the vehicle is reached (or nearly reached). It returns to the depot and start over with the remaining customers with a new vehicle. 

The algorithm is described in pseudocode:
\newline

\begin{algorithm}[!ht]
	\caption{Nearest Neightbour}\label{euclid}
	\begin{algorithmic}[1]
	\Require{Set of customers}
	\Ensure{$Solution$ (solution to the CVRP instance)}
	\State $\var{j} \gets$ random node j
	\State $\var{i} = \var{j} \gets$ starting point
	\State $\var{W} = \{1, 2, ..., n\} \backslash \{j\} \gets$ set of customers minus the randomly picked
	\Function{NearestNeightbour}{{$instance, W$}}
		\While {$W \neq \varnothing$}
			\State \text{$let\  j \in W \  |\  c_{ij} = min\{c_{ij}\  |\  j \in W\}$}
			\State Shortest node by edge $(i,j)$ is added to vehicle route
			\State Set $W = W \backslash \{j\}$ $\gets$ Customer i is marked visited
			\State $i = j$ $\gets$ setting i to next in line

		\EndWhile
	\EndFunction

	\end{algorithmic}
\end{algorithm}

The Nearest Neightbour has a running time of $\mathcal{O}(n^2)$, which is a quadric function. This means, that for every execution step done by the algorithm, the input size n has to be checked: $ n * n$.  
In the developed code, a parameter has been added to perform performance testing on the algorithms. Every instance is executed by each of the algorithms, where time, cost, and numbers of nodes are saved. In Figure \ref{fig:nnanalysis} (see appendix A for data), it is shown when the input size grows, the execution time grows quadric:

\begin{figure}[H]
	\caption{Algorithm analysis of Nearest Neightbour. $X = Time$, $Y = Nodes$}
	\centering
	\includegraphics[width=1.0\textwidth]{NN_Graph_Results.png}
	\label{fig:nnanalysis}

\end{figure}

\newpage

\section{K Nearest Neightbours with a randomized twist} 

The K-Nearest Neightbour algorithm is the same as Nearest Neightbour but with a twist. The Nearest Neightbour algorithm has to downside of ending in a local optimum. This means, the solution is optimal within the set of candidate solutions. It is different from the global optimum, which is the optimal solution for all possible solutions. KNN tries to escape the local optimum by taking an unforseen step instead of always following the closest node. It starts by picking the k nearest neightbours and then by voting, it picks one of the k nearest neightbours. The vote in this case is done by randomly choosing a customer and then continue with the same approach. 

The algorithm is described in pseudocode:
\newline

\begin{algorithm}[!ht]
	\caption{K-Nearest Neightbour}\label{euclid}
	\begin{algorithmic}[1]
	\Require{Set of customers}
	\Ensure{$Solution$ (solution to the CVRP instance)}
	\State $\var{j} \gets$ random node j
	\State $\var{i} = \var{j} \gets$ starting point
	\State $\var{W} = \{1, 2, ..., n\} \backslash \{j\} \gets$ set of customers minus the randomly picked
	\State $\var{C} = \varnothing \gets$ empty set to store closest points to i
	\Function{KNearestNeightbour}{{$instance, W$}}
		\While {$W \neq \varnothing$}
			\State \text{$let\  j \in W \  |\  c_{ij} = min\{c_{ij}\  |\  j \in W\}$}
			\State Every j is added to $C$
			\State Sort $C$ ascending and take the first k customers
			\State Set j to a random pick in $C$
			\State Set $W = W \backslash \{j\}$ $\gets$ Customer j is marked visited
			\State $i = j$ $\gets$ setting i to next in line

		\EndWhile
	\EndFunction

	\end{algorithmic}
\end{algorithm}

The K-Nearest Neightbour has a running time of $\mathcal{O}(n^2 + k)$, which is a quadric function. The execution steps for the input size is the same as the nearest neightbour, but it adds the extra execution by randomly picking one of the k nearest neigtbours.
In Figure \ref{fig:knnanalysis} (see appendix B for data), it is shown when the input size grows, the execution time grows quadric:

\begin{figure}[H]
	\caption{Algorithm analysis of K-Nearest Neightbour. $X = Time$, $Y = Nodes$}
	\centering
	\includegraphics[width=1.0\textwidth]{KNN_Graph_Results.png}
	\label{fig:knnanalysis}

\end{figure}

\newpage

\section{Furthest Neightbour Cluster - Custom} 

NN and KNN takes the nearest neightbours into consideration and keeps growing by using the nearest neightbour. This can lead to issues with the later vehicles only having the customers left, which are far away from each other. Depending on the instance, KNN - and especially - NN can grow in one direction, and not taking the rest of the instance into account.  
The project has tried to accommodate the issue by picking the customer furthest away from the depot. The algorithm has been named Furthest Neightbour Cluster - lacking of better naming. The algorithm picks the customer furtest away from the depot, finds the closest nodes to the picked customer (like a cluster), and uses NN to find a suitable route within this cluster. Capacity is taking into considartion when picking the closest customers.

The algorithm is described in pseudocode:

\begin{algorithm}[!ht]
	\caption{Furthest Neightbour Cluster}\label{euclid}
	\begin{algorithmic}[1]
	\Require{Set of customers}
	\Ensure{$Solution$ (solution to the CVRP instance)}
	\State $\var{j} \gets$ depot
	\State $\var{i} = \var{j} \gets$ starting point
	\State $\var{W} = \{1, 2, ..., n\} \backslash \{j\} \gets$ set of customers minus the randomly picked
	\State $\var{F} = \varnothing \gets$ empty set to store closest customer to furhtest customer
	\Function{FurthestNeightbourCluster}{{$instance, W$}}
		\While {$W \neq \varnothing$}
			\State \text{$let\  j \in W \  |\  c_{ij} = max\{c_{ij}\  |\  j \in W\}$}
			\State Furthest node by edge $(i,j)$ is stored in j
			\State Set $W = W \backslash \{j\}$ $\gets$ Customer j is marked visited
			\While {$W \neq \varnothing$ \& Capacity not reached}
				\State \text{$let\  k \in W \  |\  c_{jk} = min\{c_{jk}\  |\  k \in W\}$}
				\State Shortest node by edge $(j,k)$ is stored in k
				\State Set $W = W \backslash \{k\}$ $\gets$ Customer k is marked visited
				\State $i = depot$ as starting point
			\EndWhile
		\EndWhile
	\EndFunction

	\end{algorithmic}
\end{algorithm}

The Furthest Neightbour Cluster has a running time of $\mathcal{O}(3n^2)$, which is a quadric function. We can calculate it by looking into the code and analyze. The first While-loop takes up $\mathcal{O}(n)$ and the For-loop inside it takes up $\mathcal{O}(n)$. Then the While-loop inside the outer While-loop takes up the same as before: $\mathcal{O}(n)$ and the same with the For-loop inside: $\mathcal{O}(n)$. The last While- and For-loop takes up the same: $\mathcal{O}(n)$. This gives us the following: ($\mathcal{O}(n)$ * $\mathcal{O}(n)$) + ($\mathcal{O}(n)$) * $\mathcal{O}(n)$) + ($\mathcal{O}(n)$ * $\mathcal{O}(n)$) $=$ $\mathcal{O}(3n^2)$. \\
In Appendix D, the code for the algorithm is shown.\\
In Figure \ref{fig:fncanalysis} (see appendix E for data), it is shown when the input size grows, the execution time grows quadric:

\begin{figure}[H]
	\caption{Algorithm analysis of K-Nearest Neightbour. $X = Time$, $Y = Nodes$}
	\centering
	\includegraphics[width=1.0\textwidth]{FNC_Graph_Results.png}
	\label{fig:fncanalysis}

\end{figure}

\section{Two Opt \& Three Opt}

Heuristics generates solutions which can be global optimal, but tends to end up in a local optimum. One way of getting out of this is by using Local Search Optimization algorithms. These algorithms takes an existing solution to an optimization problem and try to optimize the solution. In this project, the goal is to minimize the total cost of the vehicle routes, and we can use Two Opt and Three Opt to try improving our existing solutions. \\
Two opt takes two routes and compare their distances. A swap is possible if the distance is being reduced in at least one route, while still preserving the maximum capacity limit of both routes. \\
Three Opt does the same as Two Opt, but instead of looking and two routes, it takes three into consideration. \\

Using the Nearest Neightbour heuristics to analysise the cost reduction using local search showed an improvement of $171424 - 167016 = 4408$ (sum of total costs from .csv files / see appendix F for results) over all of the instances. And this is by only running Two Opt once on every solution of each instance.

\section{Closing Notes}

Heuristics give fast and good results. They are not global optimum but if the objective is to find a solution, and not the best, they are good for this. To get closer to global optimum, local search algorithms tries to get us closer. The issue with local search algorithms are: when to stop improving. Shown in the above results, Two Opt was only ran once to improve existing solutions, but it could have been run continiously and set a kill switch, when it is no longer improving. One issue with this is, when it gets stuck in a local optimum. In that case, other approaches should be taken to try and get away from the local optimum. An example of this could be by making bad moves. 

\newpage
\begin{appendices}

\section{Nearest Neightbour Results}

\begin{table}[]
	\begin{tabular}{llll}
	Instance & Cost & Time & Total Nodes \\
	Golden\_04.xml & 15746.0 & 0.26 & 481 \\
	Golden\_13.xml & 980.0 & 0.1 & 253 \\
	Golden\_17.xml & 924.0 & 0.08 & 241 \\
	Golden\_19.xml & 1756.0 & 0.15 & 361 \\
	Golden\_05.xml & 9443.0 & 0.07 & 201 \\
	Golden\_09.xml & 575.0 & 0.08 & 256 \\
	Golden\_01.xml & 5824.0 & 0.07 & 241 \\
	Golden\_18.xml & 1413.0 & 0.13 & 301 \\
	Golden\_15.xml & 1545.0 & 0.17 & 397 \\
	Golden\_20.xml & 2288.0 & 0.21 & 421 \\
	Golden\_07.xml & 11380.0 & 0.14 & 361 \\
	Golden\_14.xml & 1281.0 & 0.12 & 321 \\
	Golden\_16.xml & 1897.0 & 0.26 & 481 \\
	Golden\_10.xml & 764.0 & 0.13 & 324 \\
	Golden\_11.xml & 974.0 & 0.17 & 400 \\
	Golden\_03.xml & 12054.0 & 0.19 & 401 \\
	Golden\_08.xml & 12876.0 & 0.23 & 441 \\
	Golden\_06.xml & 9689.0 & 0.13 & 281 \\
	Golden\_12.xml & 1165.0 & 0.27 & 484 \\
	Golden\_02.xml & 9245.0 & 0.22 & 321 \\
	A-n63-k10.xml & 1505.0 & 0.02 & 63 \\
	A-n61-k09.xml & 1294.0 & 0.01 & 61 \\
	A-n37-k05.xml & 965.0 & 0.01 & 37 \\
	A-n54-k07.xml & 1690.0 & 0.01 & 54 \\
	A-n69-k09.xml & 1555.0 & 0.02 & 69 \\
	A-n62-k08.xml & 1486.0 & 0.01 & 62 \\
	A-n53-k07.xml & 1270.0 & 0.01 & 53 \\
	A-n38-k05.xml & 966.0 & 0.0 & 38 \\
	A-n65-k09.xml & 1422.0 & 0.01 & 65 \\
	A-n60-k09.xml & 1583.0 & 0.01 & 60 \\
	A-n64-k09.xml & 1715.0 & 0.01 & 64 \\
	A-n45-k07.xml & 1614.0 & 0.0 & 45 \\
	A-n63-k09.xml & 2056.0 & 0.01 & 63 \\
	A-n39-k06.xml & 1054.0 & 0.0 & 39 \\
	A-n33-k05.xml & 765.0 & 0.0 & 33 \\
	A-n34-k05.xml & 969.0 & 0.0 & 34 \\
	A-n33-k06.xml & 903.0 & 0.0 & 33 \\
	A-n39-k05.xml & 1076.0 & 0.0 & 39 \\
	A-n46-k07.xml & 1159.0 & 0.0 & 46 \\
	A-n45-k06.xml & 1158.0 & 0.0 & 45 \\
	A-n37-k06.xml & 1164.0 & 0.0 & 37 \\
	A-n55-k09.xml & 1392.0 & 0.0 & 55 \\
	A-n44-k06.xml & 1251.0 & 0.0 & 44 \\
	A-n36-k05.xml & 1059.0 & 0.0 & 36 \\
	A-n48-k07.xml & 1316.0 & 0.01 & 48 \\
	A-n80-k10.xml & 2054.0 & 0.01 & 80 \\
	A-n32-k05.xml & 926.0 & 0.0 & 32 \\
	P-n051-k10.xml & 962.0 & 0.0 & 51 \\
	P-n101-k04.xml & 962.0 & 0.02 & 101 \\
	P-n050-k10.xml & 807.0 & 0.0 & 50 \\
	P-n023-k08.xml & 671.0 & 0.0 & 23 \\
	P-n019-k02.xml & 272.0 & 0.0 & 19 \\
	P-n045-k05.xml & 674.0 & 0.0 & 45 \\
	P-n020-k02.xml & 293.0 & 0.0 & 20 \\
	P-n060-k10.xml & 878.0 & 0.01 & 60 \\
	P-n022-k02.xml & 289.0 & 0.0 & 22 \\
	P-n050-k07.xml & 663.0 & 0.0 & 50 \\
	P-n055-k08.xml & 668.0 & 0.0 & 55 \\
	P-n070-k10.xml & 991.0 & 0.02 & 70 \\
	P-n040-k05.xml & 576.0 & 0.0 & 40 \\
	P-n016-k08.xml & 497.0 & 0.0 & 16 \\
	P-n076-k04.xml & 773.0 & 0.01 & 76 \\
	P-n060-k15.xml & 1127.0 & 0.01 & 60 \\
	P-n076-k05.xml & 839.0 & 0.02 & 76 \\
	P-n021-k02.xml & 262.0 & 0.0 & 21 \\
	P-n022-k08.xml & 690.0 & 0.0 & 22 \\
	P-n065-k10.xml & 1017.0 & 0.01 & 65 \\
	P-n055-k07.xml & 744.0 & 0.01 & 55 \\
	P-n050-k08.xml & 711.0 & 0.0 & 50 \\
	P-n055-k15.xml & 1095.0 & 0.01 & 55 \\
	P-n055-k10.xml & 791.0 & 0.01 & 55 \\
	CMT06.xml & 746.0 & 0.0 & 51 \\
	CMT05.xml & 1690.0 & 0.05 & 200 \\
	CMT04.xml & 1446.0 & 0.04 & 151 \\
	CMT07.xml & 1050.0 & 0.02 & 76 \\
	CMT03.xml & 1047.0 & 0.02 & 101 \\
	CMT01.xml & 746.0 & 0.0 & 51 \\
	CMT14.xml & 1130.0 & 0.02 & 101 \\
	CMT12.xml & 1130.0 & 0.01 & 101 \\
	CMT02.xml & 1050.0 & 0.01 & 76 \\
	CMT08.xml & 1047.0 & 0.01 & 101 \\
	CMT11.xml & 1384.0 & 0.02 & 121 \\
	CMT10.xml & 1690.0 & 0.05 & 200 \\
	CMT13.xml & 1384.0 & 0.04 & 121 \\
	CMT09.xml & 1446.0 & 0.03 & 151
	\end{tabular}
	\end{table}
\newpage


\section{K-Nearest Neightbour Results}

\begin{table}[]
	\begin{tabular}{llll}
	Instance & Cost & Time & Total Nodes \\
	Golden\_04.xml & 27923.0 & 0.42 & 481 \\
	Golden\_13.xml & 1411.0 & 0.13 & 253 \\
	Golden\_17.xml & 1207.0 & 0.09 & 241 \\
	Golden\_19.xml & 2299.0 & 0.18 & 361 \\
	Golden\_05.xml & 13912.0 & 0.08 & 201 \\
	Golden\_09.xml & 965.0 & 0.09 & 256 \\
	Golden\_01.xml & 10802.0 & 0.07 & 241 \\
	Golden\_18.xml & 1686.0 & 0.14 & 301 \\
	Golden\_15.xml & 2165.0 & 0.19 & 397 \\
	Golden\_20.xml & 3025.0 & 0.24 & 421 \\
	Golden\_07.xml & 19552.0 & 0.16 & 361 \\
	Golden\_14.xml & 1773.0 & 0.13 & 321 \\
	Golden\_16.xml & 2633.0 & 0.29 & 481 \\
	Golden\_10.xml & 1231.0 & 0.13 & 324 \\
	Golden\_11.xml & 1538.0 & 0.19 & 400 \\
	Golden\_03.xml & 20616.0 & 0.22 & 401 \\
	Golden\_08.xml & 23087.0 & 0.24 & 441 \\
	Golden\_06.xml & 17759.0 & 0.11 & 281 \\
	Golden\_12.xml & 1920.0 & 0.3 & 484 \\
	Golden\_02.xml & 15024.0 & 0.14 & 321 \\
	A-n63-k10.xml & 2060.0 & 0.01 & 63 \\
	A-n61-k09.xml & 1742.0 & 0.01 & 61 \\
	A-n37-k05.xml & 1170.0 & 0.0 & 37 \\
	A-n54-k07.xml & 1860.0 & 0.0 & 54 \\
	A-n69-k09.xml & 1755.0 & 0.01 & 69 \\
	A-n62-k08.xml & 2189.0 & 0.01 & 62 \\
	A-n53-k07.xml & 1616.0 & 0.01 & 53 \\
	A-n38-k05.xml & 1202.0 & 0.0 & 38 \\
	A-n65-k09.xml & 1882.0 & 0.01 & 65 \\
	A-n60-k09.xml & 2285.0 & 0.01 & 60 \\
	A-n64-k09.xml & 2292.0 & 0.01 & 64 \\
	A-n45-k07.xml & 1585.0 & 0.0 & 45 \\
	A-n63-k09.xml & 2487.0 & 0.01 & 63 \\
	A-n39-k06.xml & 1441.0 & 0.0 & 39 \\
	A-n33-k05.xml & 1205.0 & 0.0 & 33 \\
	A-n34-k05.xml & 1397.0 & 0.0 & 34 \\
	A-n33-k06.xml & 1107.0 & 0.0 & 33 \\
	A-n39-k05.xml & 1289.0 & 0.0 & 39 \\
	A-n46-k07.xml & 1453.0 & 0.0 & 46 \\
	A-n45-k06.xml & 1613.0 & 0.0 & 45 \\
	A-n37-k06.xml & 1509.0 & 0.0 & 37 \\
	A-n55-k09.xml & 1812.0 & 0.01 & 55 \\
	A-n44-k06.xml & 1503.0 & 0.0 & 44 \\
	A-n36-k05.xml & 1307.0 & 0.0 & 36 \\
	A-n48-k07.xml & 1717.0 & 0.0 & 48 \\
	A-n80-k10.xml & 2732.0 & 0.01 & 80 \\
	A-n32-k05.xml & 1282.0 & 0.0 & 32 \\
	P-n051-k10.xml & 1146.0 & 0.0 & 51 \\
	P-n101-k04.xml & 1353.0 & 0.01 & 101 \\
	P-n050-k10.xml & 1083.0 & 0.0 & 50 \\
	P-n023-k08.xml & 695.0 & 0.0 & 23 \\
	P-n019-k02.xml & 356.0 & 0.0 & 19 \\
	P-n045-k05.xml & 797.0 & 0.0 & 45 \\
	P-n020-k02.xml & 401.0 & 0.0 & 20 \\
	P-n060-k10.xml & 1132.0 & 0.01 & 60 \\
	P-n022-k02.xml & 371.0 & 0.0 & 22 \\
	P-n050-k07.xml & 956.0 & 0.0 & 50 \\
	P-n055-k08.xml & 1025.0 & 0.0 & 55 \\
	P-n070-k10.xml & 1310.0 & 0.01 & 70 \\
	P-n040-k05.xml & 695.0 & 0.0 & 40 \\
	P-n016-k08.xml & 537.0 & 0.0 & 16 \\
	P-n076-k04.xml & 1036.0 & 0.01 & 76 \\
	P-n060-k15.xml & 1451.0 & 0.01 & 60 \\
	P-n076-k05.xml & 1142.0 & 0.01 & 76 \\
	P-n021-k02.xml & 309.0 & 0.0 & 21 \\
	P-n022-k08.xml & 792.0 & 0.0 & 22 \\
	P-n065-k10.xml & 1279.0 & 0.01 & 65 \\
	P-n055-k07.xml & 926.0 & 0.01 & 55 \\
	P-n050-k08.xml & 1006.0 & 0.0 & 50 \\
	P-n055-k15.xml & 1308.0 & 0.0 & 55 \\
	P-n055-k10.xml & 1004.0 & 0.0 & 55 \\
	CMT06.xml & 836.0 & 0.01 & 51 \\
	CMT05.xml & 2392.0 & 0.07 & 200 \\
	CMT04.xml & 1909.0 & 0.04 & 151 \\
	CMT07.xml & 1192.0 & 0.01 & 76 \\
	CMT03.xml & 1412.0 & 0.01 & 101 \\
	CMT01.xml & 913.0 & 0.0 & 51 \\
	CMT14.xml & 1579.0 & 0.01 & 101 \\
	CMT12.xml & 1528.0 & 0.01 & 101 \\
	CMT02.xml & 1383.0 & 0.01 & 76 \\
	CMT08.xml & 1467.0 & 0.02 & 101 \\
	CMT11.xml & 1983.0 & 0.02 & 121 \\
	CMT10.xml & 2372.0 & 0.05 & 200 \\
	CMT13.xml & 1916.0 & 0.02 & 121 \\
	CMT09.xml & 1880.0 & 0.03 & 151
	\end{tabular}
	\end{table}
\newpage

\section{Furthest Neightbour Cluster - Results}

\begin{table}[]
	\begin{tabular}{llll}
	Instance & Cost & Time & Total Nodes \\
	Golden\_04.xml & 74042.0 & 0.19 & 481 \\
	Golden\_13.xml & 3137.0 & 0.05 & 253 \\
	Golden\_17.xml & 3050.0 & 0.07 & 241 \\
	Golden\_19.xml & 6218.0 & 0.1 & 361 \\
	Golden\_05.xml & 36673.0 & 0.03 & 201 \\
	Golden\_09.xml & 2521.0 & 0.07 & 256 \\
	Golden\_01.xml & 21912.0 & 0.05 & 241 \\
	Golden\_18.xml & 4350.0 & 0.07 & 301 \\
	Golden\_15.xml & 5432.0 & 0.15 & 397 \\
	Golden\_20.xml & 8151.0 & 0.14 & 421 \\
	Golden\_07.xml & 49258.0 & 0.1 & 361 \\
	Golden\_14.xml & 4211.0 & 0.11 & 321 \\
	Golden\_16.xml & 6905.0 & 0.17 & 481 \\
	Golden\_10.xml & 3519.0 & 0.11 & 324 \\
	Golden\_11.xml & 4729.0 & 0.12 & 400 \\
	Golden\_03.xml & 51120.0 & 0.12 & 401 \\
	Golden\_08.xml & 54002.0 & 0.16 & 441 \\
	Golden\_06.xml & 43663.0 & 0.07 & 281 \\
	Golden\_12.xml & 6295.0 & 0.18 & 484 \\
	Golden\_02.xml & 34681.0 & 0.1 & 321 \\
	A-n63-k10.xml & 3343.0 & 0.01 & 63 \\
	A-n61-k09.xml & 3200.0 & 0.01 & 61 \\
	A-n37-k05.xml & 1762.0 & 0.0 & 37 \\
	A-n54-k07.xml & 3374.0 & 0.0 & 54 \\
	A-n69-k09.xml & 3756.0 & 0.0 & 69 \\
	A-n62-k08.xml & 4130.0 & 0.0 & 62 \\
	A-n53-k07.xml & 2817.0 & 0.0 & 53 \\
	A-n38-k05.xml & 1982.0 & 0.0 & 38 \\
	A-n65-k09.xml & 3420.0 & 0.0 & 65 \\
	A-n60-k09.xml & 3699.0 & 0.01 & 60 \\
	A-n64-k09.xml & 3872.0 & 0.0 & 64 \\
	A-n45-k07.xml & 2592.0 & 0.0 & 45 \\
	A-n63-k09.xml & 4233.0 & 0.0 & 63 \\
	A-n39-k06.xml & 2262.0 & 0.0 & 39 \\
	A-n33-k05.xml & 1970.0 & 0.0 & 33 \\
	A-n34-k05.xml & 2076.0 & 0.0 & 34 \\
	A-n33-k06.xml & 1608.0 & 0.0 & 33 \\
	A-n39-k05.xml & 1942.0 & 0.0 & 39 \\
	A-n46-k07.xml & 2592.0 & 0.0 & 46 \\
	A-n45-k06.xml & 2617.0 & 0.0 & 45 \\
	A-n37-k06.xml & 2217.0 & 0.0 & 37 \\
	A-n55-k09.xml & 3250.0 & 0.0 & 55 \\
	A-n44-k06.xml & 2345.0 & 0.0 & 44 \\
	A-n36-k05.xml & 1990.0 & 0.0 & 36 \\
	A-n48-k07.xml & 2639.0 & 0.0 & 48 \\
	A-n80-k10.xml & 4686.0 & 0.01 & 80 \\
	A-n32-k05.xml & 2018.0 & 0.0 & 32 \\
	P-n051-k10.xml & 1814.0 & 0.01 & 51 \\
	P-n101-k04.xml & 3595.0 & 0.01 & 101 \\
	P-n050-k10.xml & 1798.0 & 0.0 & 50 \\
	P-n023-k08.xml & 733.0 & 0.0 & 23 \\
	P-n019-k02.xml & 452.0 & 0.0 & 19 \\
	P-n045-k05.xml & 1583.0 & 0.0 & 45 \\
	P-n020-k02.xml & 488.0 & 0.0 & 20 \\
	P-n060-k10.xml & 2161.0 & 0.0 & 60 \\
	P-n022-k02.xml & 571.0 & 0.0 & 22 \\
	P-n050-k07.xml & 1612.0 & 0.0 & 50 \\
	P-n055-k08.xml & 1759.0 & 0.0 & 55 \\
	P-n070-k10.xml & 2623.0 & 0.0 & 70 \\
	P-n040-k05.xml & 1301.0 & 0.0 & 40 \\
	P-n016-k08.xml & 584.0 & 0.0 & 16 \\
	P-n076-k04.xml & 2745.0 & 0.0 & 76 \\
	P-n060-k15.xml & 2035.0 & 0.0 & 60 \\
	P-n076-k05.xml & 2790.0 & 0.01 & 76 \\
	P-n021-k02.xml & 560.0 & 0.0 & 21 \\
	P-n022-k08.xml & 763.0 & 0.0 & 22 \\
	P-n065-k10.xml & 2250.0 & 0.0 & 65 \\
	P-n055-k07.xml & 1798.0 & 0.0 & 55 \\
	P-n050-k08.xml & 1508.0 & 0.0 & 50 \\
	P-n055-k15.xml & 1987.0 & 0.0 & 55 \\
	P-n055-k10.xml & 1736.0 & 0.0 & 55 \\
	CMT06.xml & 1715.0 & 0.0 & 51 \\
	CMT05.xml & 6874.0 & 0.03 & 200 \\
	CMT04.xml & 5430.0 & 0.02 & 151 \\
	CMT07.xml & 2732.0 & 0.01 & 76 \\
	CMT03.xml & 3534.0 & 0.01 & 101 \\
	CMT01.xml & 1715.0 & 0.0 & 51 \\
	CMT14.xml & 3232.0 & 0.01 & 101 \\
	CMT12.xml & 3232.0 & 0.02 & 101 \\
	CMT02.xml & 2732.0 & 0.01 & 76 \\
	CMT08.xml & 3534.0 & 0.01 & 101 \\
	CMT11.xml & 5461.0 & 0.01 & 121 \\
	CMT10.xml & 6874.0 & 0.03 & 200 \\
	CMT13.xml & 5461.0 & 0.01 & 121 \\
	CMT09.xml & 5430.0 & 0.02 & 151
	\end{tabular}
	\end{table}
\newpage


\section{Furthest Neightbour Cluster - Code}

\includegraphics[width=1.0\textwidth]{FNC_code.png}


\section{\\Two Opt - Local Search Algorithm}

\begin{algorithm}[!ht]
	\caption{Two\_Opt}
	\begin{algorithmic}[1]
		\Require{$dist(A,B)$ returns distance from point A to B}
	\Function{twoopt}{{$route\_i, route\_j, i\_to\_j, j\_to\_i$}}

	\ForAll{$i$, $j$ such that $i$, $j$ are routes index combination}
	\Statex $\text{evaluate d1 = total length of the two edges before being swapped}$
	\Statex $\text{evaluate d2 = total length of the two edges after being swapped}$

	\If{$\var{d1} > {d2}$}
		\Statex $\text{Make the swap between two the routes}$
	\Else
		\Statex $\text{return}$
	\EndIf
	\EndFor
	\EndFunction

	\end{algorithmic}
\end{algorithm}
\newpage

\section{Nearest Neightbour with Two Opt - Results}

\begin{table}[]
	\begin{tabular}{llll}
	Instance & Cost & Time & Total Nodes \\
	P-n016-k08.xml & 497 & 0 & 16 \\
	P-n019-k02.xml & 247 & 0 & 19 \\
	P-n020-k02.xml & 285 & 0 & 20 \\
	P-n021-k02.xml & 219 & 0 & 21 \\
	P-n022-k02.xml & 238 & 0 & 22 \\
	P-n022-k08.xml & 690 & 0 & 22 \\
	P-n023-k08.xml & 669 & 0 & 23 \\
	A-n32-k05.xml & 911 & 0 & 32 \\
	A-n33-k05.xml & 749 & 0 & 33 \\
	A-n33-k06.xml & 890 & 0 & 33 \\
	A-n34-k05.xml & 943 & 0 & 34 \\
	A-n36-k05.xml & 919 & 0 & 36 \\
	A-n37-k05.xml & 931 & 0 & 37 \\
	A-n37-k06.xml & 1111 & 0 & 37 \\
	A-n38-k05.xml & 954 & 0 & 38 \\
	A-n39-k06.xml & 1018 & 0 & 39 \\
	A-n39-k05.xml & 996 & 0 & 39 \\
	P-n040-k05.xml & 542 & 0 & 40 \\
	A-n44-k06.xml & 1201 & 0 & 44 \\
	A-n45-k07.xml & 1485 & 0 & 45 \\
	A-n45-k06.xml & 1104 & 0 & 45 \\
	P-n045-k05.xml & 633 & 0 & 45 \\
	A-n46-k07.xml & 1107 & 0 & 46 \\
	A-n48-k07.xml & 1277 & 0 & 48 \\
	P-n050-k10.xml & 791 & 0 & 50 \\
	P-n050-k07.xml & 650 & 0 & 50 \\
	P-n050-k08.xml & 705 & 0.01 & 50 \\
	P-n051-k10.xml & 942 & 0 & 51 \\
	CMT06.xml & 722 & 0 & 51 \\
	CMT01.xml & 722 & 0.01 & 51 \\
	A-n53-k07.xml & 1227 & 0 & 53 \\
	A-n54-k07.xml & 1554 & 0 & 54 \\
	A-n55-k09.xml & 1324 & 0 & 55 \\
	P-n055-k08.xml & 649 & 0.01 & 55 \\
	P-n055-k07.xml & 700 & 0.01 & 55 \\
	P-n055-k15.xml & 1079 & 0.01 & 55 \\
	P-n055-k10.xml & 761 & 0 & 55 \\
	A-n60-k09.xml & 1528 & 0.01 & 60 \\
	P-n060-k10.xml & 846 & 0 & 60 \\
	P-n060-k15.xml & 1119 & 0.01 & 60 \\
	A-n61-k09.xml & 1202 & 0.01 & 61 \\
	A-n62-k08.xml & 1456 & 0.01 & 62 \\
	A-n63-k10.xml & 1482 & 0 & 63 \\
	A-n63-k09.xml & 1985 & 0.01 & 63 \\
	A-n64-k09.xml & 1630 & 0.01 & 64 \\
	A-n65-k09.xml & 1331 & 0.01 & 65 \\
	P-n065-k10.xml & 992 & 0.01 & 65 \\
	A-n69-k09.xml & 1501 & 0.01 & 69 \\
	P-n070-k10.xml & 962 & 0.01 & 70 \\
	P-n076-k04.xml & 731 & 0.01 & 76 \\
	P-n076-k05.xml & 785 & 0.01 & 76 \\
	CMT07.xml & 1014 & 0.01 & 76 \\
	CMT02.xml & 1014 & 0.01 & 76 \\
	A-n80-k10.xml & 1994 & 0.01 & 80 \\
	P-n101-k04.xml & 914 & 0.02 & 101 \\
	CMT03.xml & 983 & 0.01 & 101 \\
	CMT14.xml & 1087 & 0.01 & 101 \\
	CMT12.xml & 1087 & 0.03 & 101 \\
	CMT08.xml & 983 & 0.02 & 101 \\
	CMT11.xml & 1342 & 0.02 & 121 \\
	CMT13.xml & 1342 & 0.02 & 121 \\
	CMT04.xml & 1353 & 0.03 & 151 \\
	CMT09.xml & 1353 & 0.03 & 151 \\
	CMT05.xml & 1598 & 0.05 & 200 \\
	CMT10.xml & 1598 & 0.05 & 200 \\
	Golden\_05.xml & 9127 & 0.09 & 201 \\
	Golden\_17.xml & 860 & 0.07 & 241 \\
	Golden\_01.xml & 5811 & 0.07 & 241 \\
	Golden\_13.xml & 975 & 0.09 & 253 \\
	Golden\_09.xml & 567 & 0.08 & 256 \\
	Golden\_06.xml & 9566 & 0.1 & 281 \\
	Golden\_18.xml & 1321 & 0.12 & 301 \\
	Golden\_14.xml & 1273 & 0.11 & 321 \\
	Golden\_02.xml & 9182 & 0.14 & 321 \\
	Golden\_10.xml & 754 & 0.12 & 324 \\
	Golden\_19.xml & 1656 & 0.14 & 361 \\
	Golden\_07.xml & 11273 & 0.15 & 361 \\
	Golden\_15.xml & 1537 & 0.16 & 397 \\
	Golden\_11.xml & 968 & 0.19 & 400 \\
	Golden\_03.xml & 11976 & 0.18 & 401 \\
	Golden\_20.xml & 2158 & 0.21 & 421 \\
	Golden\_08.xml & 12799 & 0.23 & 441 \\
	Golden\_04.xml & 15513 & 0.27 & 481 \\
	Golden\_16.xml & 1888 & 0.24 & 481 \\
	Golden\_12.xml & 1158 & 0.24 & 484
	\end{tabular}
	\end{table}

\end{appendices}
\end{document}